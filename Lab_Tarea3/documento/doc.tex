\documentclass[a4paper,12pt]{article}
\usepackage[utf8]{inputenc}
\usepackage[spanish,es-tabla]{babel}
\usepackage{color}
\usepackage{parskip}
\usepackage{graphicx}
\usepackage{multirow}
\usepackage{listings}
\usepackage{vmargin}
\graphicspath{ {imagenes/} }
\definecolor{mygreen}{rgb}{0,0.6,0}
\definecolor{lbcolor}{rgb}{0.9,0.9,0.9}
\usepackage{epstopdf}


\begin{document}
\title{Problemas MPI}
\author{
Christofer Fabián Chávez Carazas \\
\small{Universidad Nacional de San Agustín} \\
\small{Algoritmos Paralelos}
}

\maketitle

\section{Problema 1}

El problema de la regla del trapecio consiste en calcular el área bajo la curva dividiéndola en trapecios y calculando
su respectiva área. Esto se puede hacer de forma paralela repartiendo los trapecios entre los procesos.

\section{Problema 2}

Para probar el funcionamiento de \textit{MPI\_Scatter y MPI\_Gather} se utilizó el siguiente ejemplo: Primero se crea
un vector con números aleatorios. Se divide dicho vector entre los procesos con \textit{MPI\_Scatter}. Cada
proceso multiplica por dos todos los elementos de su vector y luego se junta el vector en el proceso 0 con \textit{MPI\_Gather}

\section{Problema 3}

El problema de la multiplicación de una matriz con un vector se puede resolver sin usar la función \textit{MPI\_Allgather},
pero existen varios métodos matemáticos que se basan en iteraciones para llegar al resultado, haciendo que el vector resultante
de la iteración actual sea el vector que se multiplica en la iteración siguiente. Para estos problemas si sería útil
utilizar \textit{MPI\_Allgather} para que todos los procesos tengan el vector que van a multiplicar en la siguiente iteración.

\section{Problema 4}

Los resultados del tiempo de ejecución, los speedups y la eficiencia de la multiplicación matriz-vector se muestran
en las Tablas \ref{tab:1}, \ref{tab:2}, \ref{tab:3} respectivamente

\begin{table}
\begin{center}
\begin{tabular}{|c|c|c|c|c|c|}\hline
comm\_sz & 1024 & 2048 & 4096 & 8192 & 16384\\\hline
1 & 4.446697 & 16.7391302 & 66.0467146 & 264.5243646 & 1050.9673596\\\hline
2 & 2.483177 & 9.6812724 & 39.4764902 & 160.9163284 & 621.1693764\\\hline
4 & 2.3790836 & 9.7565648 & 38.4766 & 152.342 & 609.9132\\\hline
8 & 1.1930944 & 4.8534 & 19.1686 & 76.5276 & 303.3648\\\hline
16 & 0.7364748 & 2.4656 & 9.6542 & 38.916 & 151.109\\\hline
\end{tabular}
\end{center}
\caption{Tiempos de ejecución de la multiplicación matriz-vector (En milisegundos)}
\label{tab:1}
\end{table}

\begin{table}
\begin{center}
\begin{tabular}{|c|c|c|c|c|c|}\hline
comm\_sz & 1024 & 2048 & 4096 & 8192 & 16384\\\hline
1 & 1 & 1 & 1 & 1 & 1\\\hline
2 & 1.7907289734 & 1.7290217141 & 1.6730645066 & 1.6438627902 & 1.6919175341\\\hline
4 & 1.8690797583 & 1.7156786782 & 1.7165423816 & 1.7363850061 & 1.7231425055\\\hline
8 & 3.7270286408 & 3.4489492315 & 3.4455679914 & 3.4565877487 & 3.4643681785\\\hline
16 & 6.0378128349 & 6.7890696788 & 6.8412415943 & 6.7973163891 & 6.955028222\\\hline
\end{tabular}
\end{center}
\caption{Speedups de la multiplicación matriz-vector}
\label{tab:2}
\end{table}

\begin{table}
\begin{center}
\begin{tabular}{|c|c|c|c|c|c|}\hline
comm\_sz & 1024 & 2048 & 4096 & 8192 & 16384\\\hline
1 & 1 & 1 & 1 & 1 & 1\\\hline
2 & 0.8953644867 & 0.8645108571 & 0.8365322533 & 0.8219313951 & 0.8459587671\\\hline
4 & 0.4672699396 & 0.4289196696 & 0.4291355954 & 0.4340962515 & 0.4307856264\\\hline
8 & 0.4658785801 & 0.4311186539 & 0.4306959989 & 0.4320734686 & 0.4330460223\\\hline
16 & 0.3773633022 & 0.4243168549 & 0.4275775996 & 0.4248322743 & 0.4346892639\\\hline
\end{tabular}
\end{center}
\caption{Eficiencia de la multiplicación matriz-vector}
\label{tab:3}
\end{table}



\end{document}
